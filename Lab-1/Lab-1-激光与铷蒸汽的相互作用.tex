\documentclass[a4paper,10pt]{article}
\usepackage[UTF8]{ctex}
\usepackage[margin=.75in]{geometry}
\usepackage{amsmath, amssymb, amsthm, bm, mathrsfs}
\usepackage{graphicx}
\usepackage{subfigure}
\begin{document}
\title{激光与铷蒸汽的相互作用}
\author{陈稼霖\and 李正阳\and 吴杨\and 薛加民}
\date{2020 年 10 月 7 日}
\maketitle
\begin{abstract}
    铷蒸汽对透过其传播的光具有两方面的作用:一方面,吸收特定频率的光;另一方面,给光带来相位差. 铷蒸汽折射率的虚部和实部分别刻画了这两方面的作用,该两者为入射光频率的函数并遵循Kramers-Kronig关系. 我们利用饱和吸收法测量了铷蒸汽中$^{85}$Rb和$^{87}$Rb两种同位素的在$780$ nm波长附近、达到超精细结构量级分辨率的吸收谱,并用Mach-Zehnder干涉仪验证了其折射率满足的Kramer-Kronig关系.
\end{abstract}

\end{document}