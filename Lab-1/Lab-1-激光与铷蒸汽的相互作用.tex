\documentclass[a4paper, 10pt]{article}
\usepackage[UTF8]{ctex}
\usepackage[margin=.75in]{geometry}
\usepackage{multicol}
\usepackage{amsmath, amssymb, amsthm, bm, mathrsfs}
\usepackage{graphicx}
\usepackage{subfigure}
%%%% subsubsubsection定义
\usepackage{titlesec}
\titleclass{\subsubsubsection}{straight}[\subsection]
\newcounter{subsubsubsection}[subsubsection]
\renewcommand\thesubsubsubsection{\thesubsubsection.\arabic{subsubsubsection}}
\renewcommand\theparagraph{\thesubsubsubsection.\arabic{paragraph}}
\titleformat{\subsubsubsection}
  {\normalfont\normalsize\bfseries}{\thesubsubsubsection}{1em}{}
\titlespacing*{\subsubsubsection}
{0pt}{3.25ex plus 1ex minus .2ex}{1.5ex plus .2ex}
\makeatletter
\renewcommand\paragraph{\@startsection{paragraph}{5}{\z@}%
  {3.25ex \@plus1ex \@minus.2ex}%
  {-1em}%
  {\normalfont\normalsize\bfseries}}
\renewcommand\subparagraph{\@startsection{subparagraph}{6}{\parindent}%
  {3.25ex \@plus1ex \@minus .2ex}%
  {-1em}%
  {\normalfont\normalsize\bfseries}}
\def\toclevel@subsubsubsection{4}
\def\toclevel@paragraph{5}
\def\toclevel@paragraph{6}
\def\l@subsubsubsection{\@dottedtocline{4}{7em}{4em}}
\def\l@paragraph{\@dottedtocline{5}{10em}{5em}}
\def\l@subparagraph{\@dottedtocline{6}{14em}{6em}}
\makeatother
\setcounter{secnumdepth}{4}
%%%%
\begin{document}
\title{激光与铷蒸汽的相互作用}
\author{陈稼霖\and 李正阳\and 吴杨\and 薛加民}
\date{2020 年 10 月 7 日}
\maketitle
\begin{abstract}
    铷蒸汽对透过其传播的光具有两方面的作用:一方面,吸收特定频率的光;另一方面,给光带来相位差. 铷蒸汽折射率的虚部和实部分别刻画了这两方面的作用,该两者为入射光频率的函数并遵循Kramers-Kronig关系. 我们利用饱和吸收法测量了铷蒸汽中$^{85}$Rb和$^{87}$Rb两种同位素的在$780$ nm波长附近、达到超精细结构量级分辨率的吸收谱,并用Mach-Zehnder干涉仪验证了其折射率满足的Kramer-Kronig关系.
\end{abstract}

\begin{multicols}{2}
\section{实验思路}

\section{理论背景}

\subsection{铷原子的能级结构与光谱}

\subsubsection{铷原子的超精细能级结构}

\subsubsection{光谱增宽}

\subsection{饱和吸收法}

\subsubsection{饱和吸收法的基本原理}

\subsubsection{交叉共振现象}

\subsection{Kramers-Kronig关系及其在气体折射率中的体现}

\subsubsection{Kramers-Kronig关系}

\subsubsection{气体折射率中的Kramer-Kronig关系}

\subsubsection{经典力学图像}

\subsubsection{量子力学图像}

\subsubsection{气体折射率实部与虚部之间的关系}

\subsection{Mach-Zehnder干涉仪原理}

\section{实验测量与验证}

\subsection{利用饱和吸收法测量铷蒸汽吸收谱}

\subsubsection{实验设定}

\subsubsection{结果与分析}

\subsection{利用Mach-Zehnder干涉仪验证折射率的Kramers-Kronig关系}

\subsubsection{实验设定}

\subsubsection{结果与分析}

\section{结论}

\section{反思}

\begin{appendix}
\section{参考文献}

\end{appendix}
\end{multicols}
\end{document}